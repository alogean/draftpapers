\documentclass[a4paper,20pt,notitlepage]{article}


% Title Page
\title{Proof of concept: GigaSpaces implementation of the SkyGuide INCH system}
\author{A. Logean, G. Reimann, P. Heim}


\begin{document}
\maketitle

\begin{abstract}
This document addresses some architectural issues associated with a GigaSpaces implementation of the SkyGuide Inch system. 
\end{abstract}

\section{Motivations}
\section{Component model}

Here a brief overview of the different components used in the architecture definition:

\begin{enumerate}
 \item Cluster of 4 “message brokers”  distributed in 4 geographical locations
\begin{itemize}
 \item Zurich
 \item Geneva
 \item Bern
 \item Lugano
\end{itemize}
 \item Services
\begin{itemize}
\item Gateway services

\textit{Open questions}
\begin{itemize}
\item \textit{List of services for each location} 
\item \textit{Are there geographic dependant ?}
\item \textit{Are these services deployed within the server ?}
\end{itemize}

\item Persistence services (RDMS / MySQL)

\item Monitoring / management services
\end{itemize}

\item Client positions (Rich/Fat clients / GUIs)
\end{enumerate}


\section{Non Functional Requirements}

\subsection{Message Brokers}

\subsection{Data}
\begin{enumerate}
 \item Size of data is known and not so big, no dynamic scaling necessary;
 \begin{itemize}
  \item \textit{Question: how big are there ?}
 \end{itemize}
 \item 40 data types, 500 data items (instances of a data type)
 \item Some data should be persisted , some other not (for example data changing rapidly)
 \item Some data are critical ( like blocking runways ) some other not (wind data). Some priority on data could be defined. 
 \item Some data are geographically specific and other not
\end{enumerate}

\subsection{Services}
\subsubsection{Gateway services}
 \begin{itemize}
  \item Each gatway service are redundant. If one crach the other take over the requests (hot standby);
\item Each gatway service is a standalone application (no dependencies) and could be deployed on any server;
\item \textbf{Service management}: it should be possible to control each service via JMX (start/restart/stop ... );
\item \textbf{Service monitoring}: it should be possible to monitor the activity of each services (flow of data, some statistics, ....) cf monitoring services;
\item \textit{Question: are there some dependencies between services ? Like a Service A consumes data from Service B etc..}
 \end{itemize}

\subsubsection{Persistency service}
\begin{itemize}
 \item Data that do not change rapidly have to be persisted;
 \item By using Gigaspaces, the writing and reading in the database works in background and is handled by the space; 
 \item In a GigaSpaces architecture this service would be used to (re-)initialize a client and to generate a check-up list (if still needed). It delivers the latest relevant objects for a specific (remote) client from the space;
\end{itemize}

\subsubsection{Monitoring and Management services}

\section{GigaSpaces programming model}
Here we should described how an event based system can be build based on the space paradigm.
 
\section{Use Cases}
Here we should try to list all the possible catastrophic scenari and show for each case possible fail-over capable architectures.

\end{document}
